\documentclass[]{article}
\usepackage[spanish]{babel}
\usepackage{graphicx}

%opening
\title{\includegraphics[width=4cm]{UTN_logo.jpg}\\[2cm]Trabajo Practico 1 - Energía, Potencia, Eficiencia \textbf{Física II}}

\author{
	Sánchez, Julián Alejandro\\
	\texttt{Legajo: 413544}
	\and
	Sánchez, Julián Alejandro\\
	\texttt{Legajo: 413544}
	\and
	Sánchez, Julián Alejandro\\
	\texttt{Legajo: 413544}
	\and
	Sánchez, Julián Alejandro\\
	\texttt{Legajo: 413544}
}

\begin{document}

\maketitle

\newpage
\section{Resumen teórico básico}

\subsection{Temperatura y ley cero de la Termodinámica}
\begin{flushleft}
	\textbf{Ley cero de la Termodinámica}: Si dos objetos A y B están por separado en equilibrio térmico con un tercer objeto C, se dice que A y B están en equilibrio térmico entre sí.
\end{flushleft}
\begin{flushleft}
	\textbf{Temperatura}: Propiedad que determina si un objeto está en equilibrio térmico con otros objetos. Dos objetos en equilibrio térmico uno con el otro están a la misma temperatura.
\end{flushleft}


\subsection{Calor y Energía}
\begin{flushleft}
	\textbf{Calor}: El calor es la transferencia de energía a través de la frontera de un sistema debido a una diferencia de temperatura entre el sistema y el ambiente.
\end{flushleft}

\begin{flushleft}
	\textbf{Energía interna}: Es toda la energía de un sistema asociada a componentes microscópicos(átomos y moléculas).
\end{flushleft}

\subsection{Calor especifico}
\begin{flushleft}
	\textbf{Capacidad térmica}: Cantidad de energía necesaria para elevar la temperatura de un muestra en 1°C, dicha energía Q produce un cambio de temperatura $\bigtriangleup T$. 
	
	\begin{center}
		$Q = C \cdot \bigtriangleup T$
	\end{center}
\end{flushleft}

\begin{flushleft}
	\textbf{Calor específico}: Es la capacidad térmica por unidad de masa.
	\begin{center}
		$c = \frac{Q}{m \cdot \bigtriangleup T}$
	\end{center}
	
	\begin{flushleft}
		Donde: \\
		\textbf{m}: Masa de una muestra de sustancia. \\
		\textbf{Q}: Energía transferida a la sustancia \textbf{m}. \\
		\textbf{$\bigtriangleup T$}: Cambio de temperatura. \\
	\end{flushleft}
	
	\begin{flushleft}
		A partir de está definición se puede relacionar la energía transferida entre una sustancia de masa \textbf{m} y sus alrededores con un cambio de temperatura $\bigtriangleup T$.
		
		\begin{center}
			$Q = m \cdot c \cdot \bigtriangleup T$
		\end{center}
	\end{flushleft}
\end{flushleft}

\subsection{Potencia}
Potencia se define como la relación que existe entra la cantidad de energía transferida desde un sistema con el tiempo requerido para realizarse dicha transferencia.

\begin{center}
	$P = \frac{E}{t}$
\end{center}
\subsection{Eficiencia}

\end{document}
